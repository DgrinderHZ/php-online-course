\documentclass[10pt,a4paper,notitlepage]{article}
%Mise en page
\usepackage[left=2cm, right=2cm, lines=45, top=0.8in, bottom=0.7in]{geometry}
\usepackage{fancyhdr}
\usepackage{fancybox}
\usepackage{pdfpages} 
\usepackage{hyperref}
\renewcommand{\headrulewidth}{1.5pt}
\renewcommand{\footrulewidth}{1.5pt}
\pagestyle{fancy}
\newcommand\Loadedframemethod{TikZ}
\usepackage[framemethod=\Loadedframemethod]{mdframed}
\usepackage{tikz}
\usetikzlibrary{calc, through, backgrounds}
\usetikzlibrary{matrix,positioning}
%Desssins geometriques
\usetikzlibrary{arrows}
\usetikzlibrary{shapes.geometric}
\usetikzlibrary{datavisualization}
\usetikzlibrary{automata} % LATEX and plain TEX
\usetikzlibrary{shapes.multipart}
\usetikzlibrary{decorations.pathmorphing} 
\usepackage{pgfplots}
\usepackage{physics}
\usepackage{titletoc}
\usepackage{mathpazo} 
\usepackage{algpseudocode}
\usepackage{algorithmicx} 
\usepackage{bohr} 
\usepackage{xlop} 
\usepackage{bbding} 
%\usepackage{minibox} 
%Ecriture arabe
\usepackage{mathdesign}
\usepackage{bbding} 
\usepackage{romande} 
\lhead{
	\textbf{ECOLE ETTI ERRACHIDIA}
}
\rhead{\textbf{Développement informatique}
}
\chead{\textbf{
		PHP : Web dynamique 
}}

\lfoot{}
\cfoot{\textbf{$\mathsf{\href{https://www.youtube.com/@zeek\_zone}{\color{blue} https://www.youtube.com/@zeek\_zone} ~~ 2023-2024}$}}
%\rfoot{\textit{Pr. $\mathcal{A}$.Kaal}}
%=====================Algo setup
\algblock{If}{EndIf}
\algcblock[If]{If}{ElsIf}{EndIf}
\algcblock{If}{Else}{EndIf}
\algrenewtext{If}{\textbf{si}}
\algrenewtext{Else}{\textbf{sinon}}
\algrenewtext{EndIf}{\textbf{finsi}}
\algrenewtext{Then}{\textbf{alors}}
\algrenewtext{While}{\textbf{tant que}}
\algrenewtext{EndWhile}{\textbf{fin tant que}}
\algrenewtext{Repeat}{\textbf{r\'ep\'eter}}
\algrenewtext{Until}{\textbf{jusqu'\`a}}
\algcblockdefx[Strange]{If}{Eeee}{Oooo}
[1]{\textbf{Eeee} "#1"}
{\textbf{Wuuuups\dots}}

\algrenewcommand\algorithmicwhile{\textbf{TantQue}}
\algrenewcommand\algorithmicdo{\textbf{Faire}}
\algrenewcommand\algorithmicend{\textbf{Fin}}
\algrenewcommand\algorithmicrequire{\textbf{Variables}}
\algrenewcommand\algorithmicensure{\textbf{Constante}}% replace ensure by constante
\algblock[block]{Begin}{End}
\newcommand\algo[1]{\textbf{algorithme} #1;}
\newcommand\vars{\textbf{variables } }
\newcommand\consts{\textbf{constantes}}
\algrenewtext{Begin}{\textbf{debut}}
\algrenewtext{End}{\textbf{fin}}
%================================
%================================

\setlength{\parskip}{1cm}
\setlength{\parindent}{1cm}
\tikzstyle{titregris} =
[draw=gray,fill=white, shading = exersicetitle, %
text=gray, rectangle, rounded corners, right,minimum height=.3cm]
\pgfdeclarehorizontalshading{exersicebackground}{100bp}
{color(0bp)=(green!40); color(100bp)=(black!5)}
\pgfdeclarehorizontalshading{exersicetitle}{100bp}
{color(0bp)=(red!40);color(100bp)=(black!5)}
\newcounter{exercise}
\renewcommand*\theexercise{exercice \textbf{Exercice}~n\arabic{exercise}}
\makeatletter
\def\mdf@@exercisepoints{}%new mdframed key:
\define@key{mdf}{exercisepoints}{%
	\def\mdf@@exercisepoints{#1}
}
\mdfdefinestyle{exercisestyle}{%
	outerlinewidth=1em,outerlinecolor=white,%
	leftmargin=-1em,rightmargin=-1em,%
	middlelinewidth=0.5pt,roundcorner=3pt,linecolor=black,
	apptotikzsetting={\tikzset{mdfbackground/.append style ={%
				shading = exersicebackground}}},
	innertopmargin=0.1\baselineskip,
	skipabove={\dimexpr0.1\baselineskip+0\topskip\relax},
	skipbelow={-0.1em},
	needspace=0.5\baselineskip,
	frametitlefont=\sffamily\bfseries,
	settings={\global\stepcounter{exercise}},
	singleextra={%
		\node[titregris,xshift=0.5cm] at (P-|O) %
		{~\mdf@frametitlefont{\theexercise}~};
		\ifdefempty{\mdf@@exercisepoints}%
		{}%
		{\node[titregris,left,xshift=-1cm] at (P)%
			{~\mdf@frametitlefont{\mdf@@exercisepoints points}~};}%
	},
	firstextra={%
		\node[titregris,xshift=1cm] at (P-|O) %
		{~\mdf@frametitlefont{\theexercise}~};
		\ifdefempty{\mdf@@exercisepoints}%
		{}%
		{\node[titregris,left,xshift=-1cm] at (P)%
			{~\mdf@frametitlefont{\mdf@@exercisepoints points}~};}%
	},
}
\makeatother


%%%%%%%%%

%%%%%%%%%%%%%%%
\mdfdefinestyle{theoremstyle}{%
	outerlinewidth=0.01em,linecolor=black,middlelinewidth=0.5pt,%
	frametitlerule=true,roundcorner=2pt,%
	apptotikzsetting={\tikzset{mfframetitlebackground/.append style={%
				shade,left color=white, right color=blue!20}}},
	frametitlerulecolor=black,innertopmargin=1\baselineskip,%green!60,
	innerbottommargin=0.5\baselineskip,
	frametitlerulewidth=0.1pt,
	innertopmargin=0.7\topskip,skipabove={\dimexpr0.2\baselineskip+0.1\topskip\relax},
	frametitleaboveskip=1pt,
	frametitlebelowskip=1pt
}
\setlength{\parskip}{0mm}
\setlength{\parindent}{10mm}
\mdtheorem[style=theoremstyle]{definition}{\textbf{Exercice}}
%================Liste definition--numList-and alphList=============
\newcounter{alphListCounter}
\newenvironment
{alphList}
{\begin{list}
		{\alph{alphListCounter})}
		{\usecounter{alphListCounter}
			\setlength{\rightmargin}{0cm}
			\setlength{\leftmargin}{0.5cm}
			\setlength{\itemsep}{0.2cm}
			\setlength{\partopsep}{0cm}
			\setlength{\parsep}{0cm}}
	}
	{\end{list}}
\newcounter{numListCounter}
\newenvironment
{numList}
{\begin{list}
		{\arabic{numListCounter})}
		{\usecounter{numListCounter}
			\setlength{\rightmargin}{0cm}
			\setlength{\leftmargin}{0.5cm}
			\setlength{\itemsep}{0cm}
			\setlength{\partopsep}{0cm}
			\setlength{\parsep}{0cm}}
	}
	{\end{list}}

\usepackage{listings,xcolor}
\usepackage{inconsolata}
\usepackage[colorlinks = true,
linkcolor = blue,
urlcolor  = blue,
anchorcolor = blue]{hyperref}
\definecolor{dkgreen}{rgb}{0,.6,0}
\definecolor{dkblue}{rgb}{0,0,.6}
\definecolor{dkyellow}{cmyk}{0,0,.8,.3}

\lstset{
	language        = php,
	basicstyle      = \small\ttfamily,
	keywordstyle    = \color{dkblue},
	stringstyle     = \color{red},
	identifierstyle = \color{dkgreen},
	commentstyle    = \color{gray},
	emph            =[1]{php},
	emphstyle       =[1]\color{black},
	emph            =[2]{if,and,or,else},
	emphstyle       =[2]\color{dkyellow},
	numbers=left,
	stepnumber=1,  
	numberfirstline=true,
	numberstyle=\footnotesize,
	xleftmargin=4.0ex,
	upquote=true,
	showlines=true, 
	rulecolor=\color{black},
}
%===========================================================
\begin{document}
	
	\begin{center}
		\large{\textbf{Contrôle N\textdegree 1: \textsc{Les éléments de base de PHP}}}
	\end{center}
	% ===================================================
	\begin{definition}
		\hspace{2ex} 
		\begin{enumerate}
			\item Parmi les variables suivantes, lesquelles ont un nom valide : \textbf{{\color{dkgreen} \$age, \$!nom, \$prix1, \$1persone }}?
			\item Déclarer une variable \textbf{{\color{dkgreen} \$nomComplet}} qui contient votre nom complet. En utilisant la fonction \textbf{{\color{violet} var\_dump()}}, afficher le type et la valeur de cette variable.
		\end{enumerate}
		Solution : \href{https://www.youtube.com/watch?v=OVFLx9-yfiw&list=PLF2W_rB6QiYBYg4-19vGs8TMTS3GbHM0H&index=1}{\color{blue} Zeek Zone}
	\end{definition}
	% ===================================================
	
	\begin{definition}
		\hspace{2ex} Donner résultats du script suivant :
		\begin{center}
			\begin{lstlisting}[language=PHP]
				<?php
				$a = "Hassan ";
				$b = "Zekkouri"
				$c = $a . $b;
				echo "a= ", $a, "b= ", $b;
				$a = 10;
				$b = 10;
				$c = $a + $b * 2;
				echo "a= ", $a, "b= ", $b, "c= ", $c;
				?> 
			\end{lstlisting}
		\end{center}
	\end{definition}
	% ===================================================

	\begin{definition}
		\hspace{2ex} 
		\begin{enumerate}
			\item Écrire un script PHP qui calcule et affiche le carré d'un nombre entier.
			\item Écrire un script PHP qui calcule l’image d’un nombre x par une fonction du type
			$f(x) = ax + b$ (a et b des variables à donner)
		\end{enumerate}
	\end{definition}
	
	
	
	% ===================================================
	\begin{definition}
		\hspace{2ex} 
		Déclarer une variable \textbf{{\color{orange} \$heure}} qui contient la valeur de type entier de votre choix comprise entre 0 et 24. 
		\begin{enumerate}
			\item Si l'heure entre [0–11], afficher : le matin
			\item Sinon Si l'heure entre [12–17], afficher : l'après-midi
			\item Sinon Si l'heure entre [18–23], afficher : la nuit
		\end{enumerate}
		
	\end{definition}
	% ===================================================
	\begin{definition}
		\hspace{2ex} 
			\begin{enumerate}
			\item Écrivez un script PHP pour afficher les nombres, de 1 à 12 avec la boucle for. 
			\item Écrivez un script  qui utilise une boucle do-while pour afficher une variable qui augmente de 2 à chaque itération, jusqu'à ce qu'elle atteigne 100.
			\item Écrivez un script PHP pour afficher des nombres de 10 à 1 en utilisant une boucle while PHP.
		\end{enumerate}
	\end{definition}
	
	\begin{center}
		Solution : \href{https://www.youtube.com/watch?v=OVFLx9-yfiw&list=PLF2W_rB6QiYBYg4-19vGs8TMTS3GbHM0H&index=1}{\color{blue} Zeek Zone}
	\end{center}
	
\end{document}