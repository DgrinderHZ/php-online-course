\documentclass[10pt,a4paper,notitlepage]{article}
%Mise en page
\usepackage[left=2cm, right=2cm, lines=45, top=0.8in, bottom=0.7in]{geometry}
\usepackage{fancyhdr}
\usepackage{fancybox}
\usepackage{pdfpages} 
\usepackage{hyperref}
\renewcommand{\headrulewidth}{1.5pt}
\renewcommand{\footrulewidth}{1.5pt}
\pagestyle{fancy}
\newcommand\Loadedframemethod{TikZ}
\usepackage[framemethod=\Loadedframemethod]{mdframed}
\usepackage{tikz}
\usetikzlibrary{calc, through, backgrounds}
\usetikzlibrary{matrix,positioning}
%Desssins geometriques
\usetikzlibrary{arrows}
\usetikzlibrary{shapes.geometric}
\usetikzlibrary{datavisualization}
\usetikzlibrary{automata} % LATEX and plain TEX
\usetikzlibrary{shapes.multipart}
\usetikzlibrary{decorations.pathmorphing} 
\usepackage{pgfplots}
\usepackage{physics}
\usepackage{titletoc}
\usepackage{mathpazo} 
\usepackage{algpseudocode}
\usepackage{algorithmicx} 
\usepackage{bohr} 
\usepackage{xlop} 
\usepackage{bbding} 
%\usepackage{minibox} 
%Ecriture arabe
\usepackage{mathdesign}
\usepackage{bbding} 
\usepackage{romande} 
\lhead{
	\textbf{ECOLE ETTI ERRACHIDIA}
}
\rhead{\textbf{Développement informatique}
}
\chead{\textbf{
		PHP : Web dynamique 
}}

\lfoot{}
\cfoot{\textbf{$\mathsf{\href{https://www.youtube.com/@zeek\_zone}{https://www.youtube.com/@zeek\_zone} ~~ 2023-2024}$}}
%\rfoot{\textit{Pr. $\mathcal{A}$.Kaal}}
%=====================Algo setup
\algblock{If}{EndIf}
\algcblock[If]{If}{ElsIf}{EndIf}
\algcblock{If}{Else}{EndIf}
\algrenewtext{If}{\textbf{si}}
\algrenewtext{Else}{\textbf{sinon}}
\algrenewtext{EndIf}{\textbf{finsi}}
\algrenewtext{Then}{\textbf{alors}}
\algrenewtext{While}{\textbf{tant que}}
\algrenewtext{EndWhile}{\textbf{fin tant que}}
\algrenewtext{Repeat}{\textbf{r\'ep\'eter}}
\algrenewtext{Until}{\textbf{jusqu'\`a}}
\algcblockdefx[Strange]{If}{Eeee}{Oooo}
[1]{\textbf{Eeee} "#1"}
{\textbf{Wuuuups\dots}}

\algrenewcommand\algorithmicwhile{\textbf{TantQue}}
\algrenewcommand\algorithmicdo{\textbf{Faire}}
\algrenewcommand\algorithmicend{\textbf{Fin}}
\algrenewcommand\algorithmicrequire{\textbf{Variables}}
\algrenewcommand\algorithmicensure{\textbf{Constante}}% replace ensure by constante
\algblock[block]{Begin}{End}
\newcommand\algo[1]{\textbf{algorithme} #1;}
\newcommand\vars{\textbf{variables } }
\newcommand\consts{\textbf{constantes}}
\algrenewtext{Begin}{\textbf{debut}}
\algrenewtext{End}{\textbf{fin}}
%================================
%================================

\setlength{\parskip}{1cm}
\setlength{\parindent}{1cm}
\tikzstyle{titregris} =
[draw=gray,fill=white, shading = exersicetitle, %
text=gray, rectangle, rounded corners, right,minimum height=.3cm]
\pgfdeclarehorizontalshading{exersicebackground}{100bp}
{color(0bp)=(green!40); color(100bp)=(black!5)}
\pgfdeclarehorizontalshading{exersicetitle}{100bp}
{color(0bp)=(red!40);color(100bp)=(black!5)}
\newcounter{exercise}
\renewcommand*\theexercise{exercice \textbf{Exercice}~n\arabic{exercise}}
\makeatletter
\def\mdf@@exercisepoints{}%new mdframed key:
\define@key{mdf}{exercisepoints}{%
	\def\mdf@@exercisepoints{#1}
}
\mdfdefinestyle{exercisestyle}{%
	outerlinewidth=1em,outerlinecolor=white,%
	leftmargin=-1em,rightmargin=-1em,%
	middlelinewidth=0.5pt,roundcorner=3pt,linecolor=black,
	apptotikzsetting={\tikzset{mdfbackground/.append style ={%
				shading = exersicebackground}}},
	innertopmargin=0.1\baselineskip,
	skipabove={\dimexpr0.1\baselineskip+0\topskip\relax},
	skipbelow={-0.1em},
	needspace=0.5\baselineskip,
	frametitlefont=\sffamily\bfseries,
	settings={\global\stepcounter{exercise}},
	singleextra={%
		\node[titregris,xshift=0.5cm] at (P-|O) %
		{~\mdf@frametitlefont{\theexercise}~};
		\ifdefempty{\mdf@@exercisepoints}%
		{}%
		{\node[titregris,left,xshift=-1cm] at (P)%
			{~\mdf@frametitlefont{\mdf@@exercisepoints points}~};}%
	},
	firstextra={%
		\node[titregris,xshift=1cm] at (P-|O) %
		{~\mdf@frametitlefont{\theexercise}~};
		\ifdefempty{\mdf@@exercisepoints}%
		{}%
		{\node[titregris,left,xshift=-1cm] at (P)%
			{~\mdf@frametitlefont{\mdf@@exercisepoints points}~};}%
	},
}
\makeatother




%%%%%%%%% 

%%%%%%%%%%%%%%%
\mdfdefinestyle{theoremstyle}{%
	outerlinewidth=0.01em,linecolor=black,middlelinewidth=0.5pt,%
	frametitlerule=true,roundcorner=2pt,%
	apptotikzsetting={\tikzset{mfframetitlebackground/.append style={%
				shade,left color=white, right color=blue!20}}},
	frametitlerulecolor=black,innertopmargin=1\baselineskip,%green!60,
	innerbottommargin=0.5\baselineskip,
	frametitlerulewidth=0.1pt,
	innertopmargin=0.7\topskip,skipabove={\dimexpr0.2\baselineskip+0.1\topskip\relax},
	frametitleaboveskip=1pt,
	frametitlebelowskip=1pt
}
\setlength{\parskip}{0mm}
\setlength{\parindent}{10mm}
\mdtheorem[style=theoremstyle]{definition}{\textbf{Exercice}}
%================Liste definition--numList-and alphList=============
\newcounter{alphListCounter}
\newenvironment
{alphList}
{\begin{list}
		{\alph{alphListCounter})}
		{\usecounter{alphListCounter}
			\setlength{\rightmargin}{0cm}
			\setlength{\leftmargin}{0.5cm}
			\setlength{\itemsep}{0.2cm}
			\setlength{\partopsep}{0cm}
			\setlength{\parsep}{0cm}}
	}
	{\end{list}}
\newcounter{numListCounter}
\newenvironment
{numList}
{\begin{list}
		{\arabic{numListCounter})}
		{\usecounter{numListCounter}
			\setlength{\rightmargin}{0cm}
			\setlength{\leftmargin}{0.5cm}
			\setlength{\itemsep}{0cm}
			\setlength{\partopsep}{0cm}
			\setlength{\parsep}{0cm}}
	}
	{\end{list}}

\usepackage{listings,xcolor}
\usepackage{inconsolata}

\definecolor{dkgreen}{rgb}{0,.6,0}
\definecolor{dkblue}{rgb}{0,0,.6}
\definecolor{dkyellow}{cmyk}{0,0,.8,.3}

\lstset{
	language        = php,
	basicstyle      = \small\ttfamily,
	keywordstyle    = \color{dkblue},
	stringstyle     = \color{red},
	identifierstyle = \color{dkgreen},
	commentstyle    = \color{gray},
	emph            =[1]{php},
	emphstyle       =[1]\color{black},
	emph            =[2]{if,and,or,else},
	emphstyle       =[2]\color{dkyellow},
	numbers=left,
	stepnumber=1,  
	numberfirstline=true,
	numberstyle=\footnotesize,
	xleftmargin=4.0ex,
	upquote=true,
	showlines=true, 
	rulecolor=\color{black},
}

\begin{document}
	
	\begin{center}
		\large{\textbf{Série N\textdegree 3: \textsc{LES BOUCLES}}}
	\end{center}
	
	% ===================================================
	Solution : \href{https://exercicesdephp.000webhostapp.com/exercices/td3.php}{\color{blue} /exercices/td3.php}
	\begin{definition}
		\hspace{2ex} 
		\begin{enumerate}
			\item Écrivez un script PHP pour afficher les nombres, de 4 à 12 en utilisant une boucle PHP. 
			\item Écrivez un script  qui utilise une boucle do-while pour afficher une variable qui augmente de 2 à chaque itération, jusqu'à ce qu'elle atteigne 10.
			\item Écrivez un script PHP pour afficher des nombres de 10 à 1 en utilisant une boucle PHP.
			\item En utilisant la boucle for, afficher la table de multiplication du chiffre 5.
			\item En utilisant deux boucles for, écrire un script qui produit le résultat ci-dessous.\\
			1\\
			22\\
			333\\
			4444\\
			55555
		\end{enumerate}
		
	\end{definition}
	% ===================================================
	\begin{definition}
		\hspace{2ex} Déclarer une variable avec la valeur 0. Tant que cette variable n'atteint pas 20, il faut :
		\begin{itemize}
			\item l'afficher ;
			\item incrémenter sa valeur de 2 ;
		\end{itemize}
		Si la valeur de la variable est égale à 10, la mettre en valeur avec la balise HTML appropriée.
	\end{definition}
	% ===================================================
	\begin{definition}
		\hspace{2ex} 
		\begin{enumerate}
			\item Écrivez un script qui utilise une boucle while pour afficher tous les nombres pairs de 1 à 50.
			
			\item Écrivez un script qui utilise une boucle do-while pour générer et afficher une suite de 5 nombres aléatoires entre 1 et 100.
			
			\item Écrivez un script qui utilise une boucle for pour calculer et afficher la factorielle d'un nombre donné.
			\begin{equation}
				n! = 1 \times2 \times 3 \times \dots \times n= \prod_{i=1}^{n}{i}
			\end{equation}
			
			\item Écrivez un script qui utilise une boucle while pour trouver et afficher la somme des carrés des nombres de 1 à 10.
			\begin{equation}
					Somme = {1}^{2}+{2}^{2}+{3}^{2}+\dots +{10}^{2}=\sum _{i=1}^{10}{i}^{2}
			\end{equation}
		\end{enumerate}
	\end{definition}
	% ===================================================
	\begin{definition}
		\hspace{2ex}  
		\begin{enumerate}
			\item Écrivez un script qui utilise une boucle for pour générer la suite de Fibonacci jusqu'à un certain nombre donné.
			\begin{equation}
				\left\{ \begin{array}{c} F_{n}=F_{n-1} + F_{n-2}\\ F_{0} = 0\\ F_{1} = 1\end{array}\right\
			\end{equation}
			
			\item Écrivez un script qui utilise une boucle while pour rechercher et afficher tous les nombres premiers inférieurs à 100.
		\end{enumerate}
	\end{definition}
	% ===================================================
	\begin{definition}
		\hspace{2ex} 
		Écrivez un script qui utilise une boucle for pour afficher un motif pyramidal de caractères, par exemple :
		\begin{lstlisting}
			   *
			  ***
			 *****
			*******
		\end{lstlisting}
		
	\end{definition}
	
	
	
	
	
	
	% ===================================================
	% \begin{definition}[]
		% \hspace{2ex}Ecrire, en utilisant la boucle \textbf{Pour}, les algorithmes qui effecturent les calculs suivants
		% \begin{enumerate}
			% \item 
			% \begin{tabular}{p{3cm}p{3cm}p{3cm}}
				% a) S = $\sum_{i=1}^{20} i$ & b) S = $\sum_{i=1}^{20} i^2$  & c) S = $ \sum_{i=1}^{20} i^i$ 
				% \end{tabular} 
			% \item
			% \begin{tabular}{p{3cm}p{3cm}p{3cm}}
				% a) P = $\prod_{k=1}^{20} k$ & b) P = $ \prod_{k=1}^{20} k^2$  &c) P = $ \prod_{k=1}^{20} k^k $\\
				% \end{tabular}
			% \end{enumerate}
		% \end{definition}
	
	% \begin{definition}
		% \hspace{2ex} Ecrire les boucles appropri\'es pour cacluler chacune des expressions ci-desosus\\
		% \begin{enumerate}
			% \item
			% \begin{tabular}{p{7cm}p{7cm}}
				% a) $s= 1^2-2^2+\dots +19^2-20^2$ & b)$ s = 1^1 - 2^2 + \dots +19^{19} -20^{20}$ \\
				% \end{tabular}
			% \item
			% \begin{tabular}{p{7cm}p{7cm}}
				% a) $s= 1^2\times (-2)^2 \times \dots \times 19^2 \times (-20)^{2}$&b)$ p = 1^1 \times 2^2 + \dots +19^{19} \times 20^{20}$ \\
				% \end{tabular}
			% \item
			% \begin{tabular}{p{7cm}p{7cm}}
				% a) $s= \sqrt{1}+\sqrt{2}+\dots +\sqrt{19}+\sqrt{20}$ &b)$ s = \dfrac{1^1}{\sqrt{2}} + \dfrac{2^2}{\sqrt{3}} + \dots +\dfrac{19^{19}}{\sqrt{20}}$ \\
				% \end{tabular}
			% \end{enumerate}
		% \end{definition}
	% \begin{definition}[]
		% \begin{minipage}{0.7\textwidth}
			% \hspace{2ex}Ex\'ecuter l'algorithme ci-contre avec les entr\'ee de la ligne 1 du tableau ci-dessous et compl\'eter la ligne 2.\\[2ex]
			% \begin{tabular}{c|c|c|c|c|c|c}
				
				% Ex\'ecution \HandRight & 1 & 2 & 3 & 4 & 5 & 6 \\ 
				% \hline 
				% $N$ & 7 & 11 & 13 & 25 & 37 & 38 \\ 
				% \hline 
				% $p$ & ... & ... & ... & ... & ... & ... \\ 
				% \hline 
				% \end{tabular}
			% \vspace{3mm}\\
			% D'apr\`es les valeurs de $N$ et de $p$, que repr\'esente la valeur de $p$. 
			% \end{minipage}
		% \begin{minipage}{0.3\textwidth}
			% \begin{scriptsize}
				% \begin{algorithmic}[1]
					% 	\State $p \leftarrow vrai$;
					% 	\State $i \leftarrow 2$;
					% 	\State Lire (N)
					% 	\Repeat 
					% 		\State $r \leftarrow Reste(N, i)$;
					% 		\If{(r==0)} \textbf{alors}
					% 		\State $p \leftarrow faux$
					% 		\EndIf
					% 		\State $i\leftarrow i+1$
					% 	\Until{(($i>=N-1$) OU ($p==faux$))}
					% \end{algorithmic}
				% \end{scriptsize}
			% \end{minipage}
		% \end{definition}
	
	% \begin{definition}[]
		% \begin{minipage}{0.7\textwidth}
			% Ex\'ecuter l'algorithme ci-contre avec les entr\'ees $a$ et $b$ des lignes 1 et 2 du tableau ci-dessous et compl\'ter la ligne 3.\\[3ex]
			% \begin{tabular}{c|c|c|c|c|c|c}
				% Ex\'ecution \HandRight & 1 & 2 & 3 & 4 & 5 & 6 \\ 
				% \hline 
				% $a$ & 2 & 3 & 13 & 25 & 37 & 16 \\ 
				% \hline 
				% $b$ & 4 & 5 & 6 & 12 & 12 & 38 \\ 
				% \hline 
				% $q$ & ... & ... & ... & ... & ... & ... \\ 
				% \hline 
				% \end{tabular} 
			
			% \vspace{3mm}
			% D'apr\`es les valeurs de $a$, $b$ et de $q$, qu'indique de la valeur de $q$ ?
			% \begin{alphList}
				% \item le maximum de $a$ et $b$,
				% \item le PGCD de $a$ et $b$,
				% \item le PPCM de $a$ et $b$.
				% \end{alphList}
			% \end{minipage}
		% \begin{minipage}{0.3\textwidth}
			% \begin{scriptsize}
				% \begin{algorithmic}[1]
					% 	\State Lire(a,b);
					% 	\State $i \leftarrow 2$;
					% 	\If{(a<b)}\textbf{}alors
					% 	\State $temp\leftarrow a$;
					% 	\State $a\leftarrow b$;
					% 	\State $b\leftarrow temp$;
					% 	\EndIf
					% 	\State $r \leftarrow Reste(a,b)$;
					% 	\While{($r<>0$)} \textbf{faire}
					% 		\State $a \leftarrow b$;
					% 		\State $b \leftarrow r$;
					% 		\State $r \leftarrow Reste(a,b)$;
					% 	\EndWhile
					% 		\State $q \leftarrow b$;	
					% \end{algorithmic}
				% \end{scriptsize}
			% \end{minipage}
		% \end{definition}
	
	% \begin{definition}[]
		% \begin{numList}
			% \item Ecrire, en utilisant une structure de contr\^ole de votre choix, un algorithme qui calcule le produit suivant
			% $$f = \prod_{k=1}^{k=n}k = k!= 1 \times 2 \times \dots \times (n-1) \times n $$
			% \item Ecrire, en utilisant une structure de contr\^ole de votre choix, un algorithme qui calcule la somme
			% $$s = \sum_{q=1}^{q=M}q! = 1! + 2! + \dots + M!$$
			% \end{numList}
		% \end{definition}
	
	
	
	%% Macros for ``successive divisions'' 
	%%
	%\def\Division#1#2#3{ % Dividend, divisor, remainder
		% \matrix (D) [matrix of nodes,
		%              below=0pt of D-1-2.south east,
		%              row sep=1pt, column sep=1pt,
		%              every node/.append style={minimum width=12mm}] {
			%   #1 \pgfmatrixnextcell #2 \\
			%   |[marcar] (R#1)| #3      \\
			% };
		% \draw[shorten >=2pt, shorten <=2pt]
		%   (D-1-2.north west) |- (D-1-2.south east);
		%}
	%\def\FinDivision#1{
		%\node[marcar, below=2pt of D-1-2.south] (C)(C)  {#1};
		%}
	%\tikzset{marcar/.style={circle,draw,inner sep=2pt,minimum width=0pt,
			%fill=yellow!10}}
	%
	%
	%\begin{tikzpicture}
	%  \coordinate (D-1-2) at (0,0) {}; % We must start with this command.
	%  \Division{25}{2}{1} % First dividend, divisor, remainder
	%  \Division{12}{2}{0} % Dividend (previous quotient), divisor, remainder
	%  \Division{6}{2}{0}  
	%  \Division{3}{2}{1}  
	%  \FinDivision{1}     % Last remainder.
	%
	%% We can draw an arrow jumping from one remainder 
	%% to the next one. Every reminder is a node called
	%% Rdividend. Last remainder is node C.
	%  \draw[shorten <=1mm, ->, dashed] (C) to[out=-150,in=-65] (R3);
	%  \draw[shorten <=1mm, ->, dashed] (R3) to[out=-150,in=-65] (R6);
	%  \draw[shorten <=1mm, ->, dashed] (R6) to[out=-150,in=-65] (R12);
	%  \draw[shorten <=1mm, ->, dashed] (R12) to[out=-150,in=-65] (R25);
	%
	%% Some more information:
	%  \node (MSB) at ([yshift=-1.3cm]R6.south) {Most significant bit (MSB)};    
	%  \node (LSB) at ([yshift=-2mm]MSB.south) {Less significant bit (LSB)}; 
	%\draw[ ->] (MSB.east) to[out=30,in=-55] (C);
	%\draw[ ->] (LSB.west) to[out=150,in=-95] (R25);
	%\end{tikzpicture}
	%\begin{center}\SnowflakeChevronBold \SnowflakeChevronBold \SnowflakeChevronBold \end{center}
	%%----------------------------------------------------------------------------------------------------------------------------------------
	%
	%%\includepdf[doublepages=true]{serie55}
	
\end{document}